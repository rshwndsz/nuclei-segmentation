\documentclass[11pt]{article}
\usepackage[utf8]{inputenc}
\usepackage{amsmath,amsthm,amsfonts,amssymb,amscd}
\usepackage{multirow,booktabs}
\usepackage[table]{xcolor}
\usepackage{fullpage}
\usepackage{lastpage}
\usepackage{enumitem}
\usepackage{fancyhdr}
\usepackage{mathrsfs}
\usepackage{wrapfig}
\usepackage{setspace}
\usepackage{calc}
\usepackage{multicol}
\usepackage{cancel}
\usepackage[retainorgcmds]{IEEEtrantools}
\usepackage[margin=3cm]{geometry}
\usepackage{amsmath}
\newlength{\tabcont}
\setlength{\parindent}{0.0in}
\setlength{\parskip}{0.05in}
\usepackage{empheq}
\usepackage{framed}
\usepackage[most]{tcolorbox}
\usepackage{xcolor}
\colorlet{shadecolor}{orange!15}
\parindent 0in
\parskip 12pt
\geometry{margin=1in, headsep=0.25in}
\theoremstyle{definition}
\newtheorem{defn}{Definition}
\newtheorem{reg}{Rule}
\newtheorem{exer}{Exercise}
\newtheorem{note}{Note}
\usepackage[superscript,biblabel]{cite}
\usepackage[nottoc]{tocbibind}

\begin{document}
  \thispagestyle{empty}
  \begin{center}
    {\LARGE \bf Summer Internship Report}\\
    {\large Russel Shawn Dsouza}\\
  \end{center}
  
  \tableofcontents
  
  \newpage
  \begin{center}
    \section*{Semantic Segmentation of H\&E stained nuclei in Histopathology images}
      \textbf{Abstract} --
      Medical imaging results in faster assessment of almost every disease from lung cancer to heart disease. By automating the analysis of medical images we can unlock even faster diagnosis of critical diseases from cancer to common-cold. The detection and segmentation of cell nuclei in slide images is a major part of most automated detection \& diagnosis methods.
  \end{center}

  \section{Introduction}
    Tissue slides are informative on many aspects of the disease such as cancer subtype, the grade or the reaction of the patient's immune system, morphological properties of cells and tissues, spatial organization of cells to name a few.
    However, interpretation of such data is not straightforward and requires special training and a deep understanding of the underlying physiology.

    The revolutionary step of tissue glass slide digitization has opened up exciting possibilities in the world of digital pathology. 
    With recent advancements in digital pathology, a high volume of quality digitized data is available for algorithm developers, scientists, and pathologists around the world.\cite{DLInDigitalPathology}

  \section{Literature Review}
    Sliding Window Approach for detection and segmentation of nuclei in H\&E images
    CNNs
    FCN\cite{Long2015FullyCN}
    Postprocessing

  \newpage
  \section{Problem Statement}
    Segment the cell nuclei from the surrounding tissue in H\&E stained Histopathology images of kidney tissues using the U-Net architecture\cite{Ronneberger2015UNetCN}.

  \section{Dataset}
    H\&E stained images of kidney tissues was used as the dataset for this project.\\
    The dataset is pre-divided into 3 sections: Train, Validate, Test.
    The Train set has 400 $400\times400\times3$ images, where 3 is the number of channels and refers to the RGB colorspace. 
    Each image has a unique ID and the corresponding ground-truth to each image is also names using the same ID. 
    The ground-truth images are of the size $400\times400$ i.e. grayscale images with only 0 or 255 intensities. 
    The Validate set has 50 images with ground-truth, the same size as those in the Train set.
    The Test set has 20 images, the same size as those in the Train set.

  \section{Architecture}
    U-Net\cite{Ronneberger2015UNetCN}
    Color deconvolution\cite{Ruifrok2001QuantificationOH}
    Color normalization

  \section{Implementation}
    \begin{itemize}
      \item PyTorch \cite{paszke2017automatic}
      \item Kaiming Initialization
      \item Data Augmentation
      \item torchvision
      \item morphological postprocessing with opencv
      \item metrics in sklearn/ignite
    \end{itemize}
  \section{Experimental results and Discussions}
  \section{Conclusion}
  \section{Future Scope}
   
  \newpage
  \addcontentsline{tocs}{section}{References}
  \bibliographystyle{ieeetr}
  \bibliography{report_references}
  \medskip
\end{document}
